% haskanoid.tex
\begin{hcarentry}[updated]{Haskanoid}
\report{Christina Zeller and Ivan Perez}%11/17
\status{open source, actively developed}
\label{haskanoid}
\makeheader

Haskanoid (Github: \href{https://git.io/v8eq3}{https://git.io/v8eq3}, Hackage:
\href{https://goo.gl/AhTUuj}{https://goo.gl/AhTUuj}, is a free/open-source
arkanoid game featuring SDL graphics and sound with Wiimote and Kinect support,
which works on Windows, Linux, Mac, Android, iOS and web browsers (thanks to
GHCJS). It is implemented using the Functional Reactive Programming library Yampa \td{refernece?}. 

%**<img width=700 src="./android.gif">
%*ignore
\begin{center}
  \includegraphics[width=.7\columnwidth]{screenshots/android.png}
  % \caption{Screenshot of Haskanoids on Android.}
\end{center}
%*endignore



% to include
% same code except web --- other backend
% accelerometers
% over 600 fps desktop
% If you want to include Haskell code, consider using lhs2tex syntax (\url{http://people.cs.uu.nl/andres/lhs2tex/}).


% (WHAT IS ITS STATUS? / WHAT HAS HAPPENED SINCE LAST TIME?)
Haskanoid is under active development. Recently, five new levels were added.

% (CAN OTHERS GET IT?)

% (WHAT ARE THE IMMEDIATE PLANS?)

% \FurtherReading
%   \url{(PROJECT URL)}

To collaborate with our research or to get further material for the use of
teaching functional programming or other subjects, please contact keera studios
\td{keera@keera.co.uk(\href{mailto:ixp@cs.nott.ac.uk}{ixp@cs.nott.ac.uk})}.

We encourage all Haskellers to participate on the development and improvement of
Haskanoid by opening issues on our Github page (\href{https://git.io/vFtZ0}{https://git.io/vFtZ0}),
adding improvements for example related to the used physics/collision subsystem,
the performance, the game design, the resource management or the supported devices. 

\end{hcarentry}
